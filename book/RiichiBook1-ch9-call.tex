%~~~~~~~~~~~~~~~~~~~~~~~~~~~~~~~~~~~~~~~~~~~~~~~~~
% Riichi Book 1, Chapter 9: Meld
%~~~~~~~~~~~~~~~~~~~~~~~~~~~~~~~~~~~~~~~~~~~~~~~~~

\chapter{Melding judgement} \label{ch:call}
\thispagestyle{empty}

\section{To meld or not to meld?}

Melding decisions --- to call {\jap pon}/{\jap chii} or not to call --- depend on a lot of variables. The most important criteria of all are the following two. 

\bigskip
\begin{itembox}[c]{When \emph{not} to meld}
Do not meld if one of the following two holds.
\be
\i The hand is \emph{both} cheap and far from ready.
\i Melding significantly reduces the hand value.
\ee
\end{itembox}

\bigskip
I will discuss each of the two in turn, and then discuss exceptional situations that justify melding even when the two conditions above are satisfied. 

\bigskip
Throughout this chapter, I assume that you are playing with rule sets that allow open {\jap tanyao}. Although older EMA rules did not allow it, revised EMA rules (effective as of April, 2016) now allow {\jap tanyao} to be an open hand.\index{european@EMA} 
I will also assume that you are the South player in the 6th turn in East-1 unless otherwise stated. 

\subsection{When not to meld 1: cheap and slow}
Melding is acceptable only when your hand is either expensive or fast. 
Melding with a cheap \emph{and} slow hand is one of the two biggest sins in riichi mahjong.\footnote{As we learned in Chapter \ref{ch:riichi}, the other big sin is meaningless {\jap dama}.} 

\bigskip
When we say a hand is ``slow'', we mean that the hand is 2-away or worse \emph{after} melding (i.e., 3-away or worse before melding) \emph{and} with a bad wait. Let's see a few examples. 

\begin{itembox}[r]{Cheap and slow hand}
\bp
\wan{3}\wan{5}\wan{7}\tong{3}\tong{4}\tong{6}\tong{6}\tong{7}\tong{9}\suo{3}\suo{5}\bai\bai~~\wan{1}\\
\hfill\footnotesize{{\jap Dora}~~~~~~~~~~}
\ep
\vspace{-20pt}An opponent discarded {\LARGE\bai} just now.
\end{itembox}
You should not call {\jap pon} in this situation. Even after calling {\jap pon}, the hand is still 2-away from ready with a couple of bad waits, as follows.
\bp
\wan{3}\wan{5}\wan{7}\tong{3}\tong{4}\tong{6}\tong{6}\tong{7}\suo{3}\suo{5}~\rbai\bai\bai~~\wan{1}\\
\hfill\footnotesize{{\jap Dora}~~~~~~~~~~~}
\ep
The probability of winning this hand any time soon is not very high. What if an opponent calls riichi now? You will have nothing but simple tiles between 3 and 7 to discard. It is not worthwhile to discard such tiles against riichi when you have a cheap 2-away hand. When your hand is cheap and slow, you should worry more about keeping safe tiles such as {\LARGE\bai} than about winning the hand.

\vfill
\begin{itembox}[r]{Slow but expensive hand}
\bp
\wan{3}\wan{5}\wan{7}\tong{3}\tong{4}\tong{6}\tong{6}\tong{7}\tong{9}\suo{3}\suo{5}\bai\bai~~\tong{6}\\
\hfill\footnotesize{{\jap Dora}~~~~~~~~~~}
\ep
\vspace{-20pt}An opponent discarded {\LARGE\bai} just now.
\end{itembox}
This time, you can call {\jap pon} on {\LARGE\bai} then discard {\LARGE\tong{9}}. This hand is still slow (after all, the hand shape is exactly the same as before), but it has a potential to be 7700 (White Dragon + {\jap sanshoku} + two {\jap dora}) even when you open it. When you see a high score potential, you can meld even with a slow hand. 

\bigskip
\begin{itembox}[r]{Cheap but fast hand}
\bp
\wan{5}\wan{6}\tong{3}\tong{4}\tong{6}\tong{6}\tong{7}\tong{9}\suo{2}\suo{3}\suo{3}\bai\bai~~\wan{4}\\
\hfill\footnotesize{{\jap Dora}~~~~~~~~~~}
\ep
\vspace{-20pt}An opponent discarded {\LARGE\bai} just now.
\end{itembox}
Again, you can call {\jap pon} on {\LARGE\bai} then discard {\LARGE\tong{9}}. This hand is cheap and it will be 2-away after the {\jap pon}, but all the remaining blocks have a good wait. When you can expect to win a hand in no time, you can meld even with a cheap hand. 

\bigskip
\color{MyRed}
\begin{itembox}[c]{Melding judgement 1}\normalcolor
Don't meld with a cheap and slow hand!
\end{itembox}\normalcolor

\bigskip
\subsection{When not to meld 2: big gap in hand values}
You should also refrain from melding when doing so significantly reduces the hand value. More specifically, do not meld when the hand value drops from 7700 or above to 2000 or below.\footnote{When the hand value reduces from {\jap haneman} to {\jap mangan}, or from {\jap baiman} to {\jap haneman}, melding is acceptable. This is because {\jap mangan} is already expensive enough.} 

\bigskip
\begin{itembox}[r]{Cheap but fast hand}
\bp
\wan{4}\wan{5}\wan{6}\wan{7}\wan{7}\tong{2}\tong{3}\tong{3}\tong{4}\tong{4}\suo{4}\suo{5}\suo{5}~~\suo{6}\\
\hfill\footnotesize{{\jap Dora}~~~~~~~~~~}
\ep
\vspace{-20pt}The left player discarded {\LARGE\suo{3}} just now.
\end{itembox}
\noindent
Do not meld with a hand like this, at least until the 13th turn or so (the third row in the discard). Although calling {\jap chii} on {\LARGE\suo{3}} will make the hand ready with a good wait, the hand value reduces to 1000. 
If you keep the hand closed and call riichi, the hand can potentially be a game-deciding hand with a realistic possibility of getting {\jap mangan} or {\jap haneman}.\footnote{{\jap Riichi} + {\jap tanyao + pinfu + iipeiko + dora} = {\jap mangan ron} or {\jap haneman tsumo}.} 

\bigskip
\begin{itembox}[r]{Fast and expensive hand}
\bp
\hspace{-173pt}{\footnotesize\color{red!75!black} Red}\\ \vspace{-16pt}
\wan{2}\wan{2}\rfw\wan{6}\wan{7}\tong{5}\tong{5}\tong{6}\suo{3}\suo{4}\suo{4}\suo{5}\suo{6}~~\suo{4}\\
\hfill\footnotesize{{\jap Dora}~~~~~~~~~~}
\ep
\vspace{-20pt}The left player discarded {\LARGE\tong{7}} just now.
\end{itembox}
\noindent
Calling {\jap chii} on {\LARGE\tong{7}} will make the hand ready with a good wait and a high score (7700). It is true that this hand can be even more expensive if you keep it closed. However, 7700 is already pretty expensive. An additional {\jap han} does not improve the hand value as much beyond 4 {\jap han}. We should thus call {\jap chii} on {\LARGE\tong{7}}, especially after the 9th turn or so. 

\bigskip
\color{MyRed}
\begin{itembox}[c]{Melding judgement 2}\normalcolor
Don't meld if melding significantly reduces the hand value!
\end{itembox}\normalcolor

\bigskip
\section{Melding choice: examples}
We will now see more examples of melding judgements, some of which will describe an exception to the two conditions introduced so far. 

\subsection{Eliminating bad waits}
One of the purposes of melding is to eliminate a bad wait in a hand to enhance speed. When you can call {\jap pon} or {\jap chii} to complete a bad-wait block in your hand it often makes sense to do so. More specifically, when you call {\jap chii} with an edge-wait or closed-wait protorun to make the hand ready, you should meld. Consider the following hand.

\begin{screen}
\bp
\wan{8}\wan{9}\tong{7}\tong{8}\tong{9}\suo{1}\suo{2}\suo{2}\suo{5}\suo{6}\suo{7}\fa\fa~~\xi\\
\hfill\footnotesize{{\jap Dora}~~~~~~~~~~}
\ep
\vspace{-20pt}The left player discarded {\LARGE\wan{7}} just now.
\end{screen}
\noindent
You should call {\jap chii} on {\LARGE\wan{7}} and discard {\LARGE\suo{1}}. It is true that doing so means that the hand value will be 1000 (Green Dragon only) and that the hand can be won only with {\LARGE\fa}. However, notice that the hand value is not very high anyway if you keep the hand closed. Even if you draw {\LARGE\wan{7}} and call riichi, the hand value is 2600 if you win on {\LARGE\fa} or 1300 if you win on {\LARGE\suo{2}}. 

\bigskip
Keep in mind also that winning a cheap hand like this is not totally meaningless.  This is because doing so also means you prevent your opponents from winning their (possibly expensive) hands. You do not want to make your {\jap mangan} hand into a 1000 hand, but the hand above is not a {\jap mangan} hand.

\bigskip
Moreover, this hand has pretty low tile acceptance (4 kinds--12 tiles, {\LARGE\wan{7}\suo{2}\suo{3}\fa}); the chance of making the hand ready without melding is not very high, either. 

\color{MyRed}
\begin{itembox}[c]{Melding judgement 3}\normalcolor
If you can eliminate a bad wait and make the hand ready, meld!
\end{itembox}\normalcolor

\subsection{Improving the wait}
It sometimes makes sense to meld even when your hand is already ready, as long as doing so improves the wait and/or the scores. It may also make sense to meld to make a bad-wait 1-away into a good-wait 1-away one. 

\bigskip
\begin{itembox}[r]{Improving the wait}
\bp
\wan{2}\wan{4}\rfw\wan{6}\wan{7}\tong{3}\tong{4}\rfd\suo{1}\suo{1}~~\zhong\zhong\rzhong
\ep
\vspace{-10pt}The left player discarded {\LARGE\wan{5}} just now.
\end{itembox}
\noindent
The hand is already ready, waiting for {\LARGE\wan{3}}. However, you should call {\jap chii} on {\LARGE\wan{5}} with {\LARGE\wan{6}\wan{7}} and discard {\LARGE\wan{2}}, so you can upgrade the wait to a side wait of {\LARGE\wan{3}-\wan{6}}. 
With melded hands, it is important to think about the possibilities of improving the wait and/or scores by melding further. 
In the current example, calling {\jap chii} on {\LARGE\wan{5}-\wan{8}} or {\jap pon} on {\LARGE\suo{1}} will improve the wait from a closed wait to a 2-way wait. 

\bigskip
\subsection{Confirming {\jap yaku}}
Ideally, we would like to complete a bad-wait block by melding so that we can have a good-wait block as the final wait of the hand. However, it sometimes makes sense to complete a good-wait block by melding if doing so confirms a certain {\jap yaku} in a hand. 

\bigskip
\begin{itembox}[r]{Confirming {\jap yaku}}
\bp
\wan{2}\wan{4}\tong{2}\tong{2}\tong{6}\tong{7}\suo{1}\suo{2}\rfs\suo{6}\suo{7}\suo{8}\suo{9}~~\suo{9}\\
\hfill\footnotesize{{\jap Dora}~~~~~~~~~~}
\ep
\vspace{-20pt}The left player discarded {\LARGE\suo{4}} just now.
\end{itembox}
\noindent
Calling {\jap chii} on {\LARGE\suo{4}} completes a side-wait block in this hand, leaving the hand 1-away with one edge-wait and one side-wait protoruns. 
However, this is acceptable because calling {\jap chii} on {\LARGE\suo{4}} confirms {\jap ittsu} in this hand. Getting {\jap ittsu} with this hand requires that we have {\LARGE\suo{4}}, not {\LARGE\suo{7}}, to complete the protorun {\LARGE\suo{5}\suo{6}}. We should thus think of this protorun more as an edge-wait protorun rather than a side-wait protorun. Calling {\jap chii} on {\LARGE\suo{4}} is tantamount to eliminating a bad wait in this case.\footnote{Note that you should discard the \wan{2}, not the \wan{4}, after melding. This is because we will discard the {\tong{6}\tong{7}} block if we draw the red \rfw.}

\bigskip
\begin{itembox}[r]{Confirming {\jap yaku}}
\bp
\tong{2}\tong{2}\tong{5}\tong{6}\tong{7}\tong{8}\suo{3}\suo{4}\dong\dong\dong\nan\nan~~\fa\\
\hfill\footnotesize{{\jap Dora}~~~~~~~~~~}
\ep \index{honitsu@{\jap honitsu} (Half Flush)}
\vspace{-20pt}The left player discarded {\LARGE\tong{2}} just now.
\end{itembox}
\noindent
If you call {\jap pon} on {\LARGE\tong{2}} and discard {\LARGE\tong{8}}, the hand is ready. However, doing so only gives you a 1300 hand. Instead, you should discard {\LARGE\suo{4}} after calling {\jap pon} to have a 1-away {\jap honitsu} hand, as follows.
\bp
\tong{5}\tong{6}\tong{7}\tong{8}\suo{3}\dong\dong\dong\nan\nan~\tong{2}\rtong{2}\tong{2}~~\fa\\
\hfill\footnotesize{{\jap Dora}~~~~~~~~~~~~}
\ep
You can further call {\jap chii} on any of {\LARGE\tong{4}\tong{6}\tong{7}\tong{9}} or call {\jap pon} on {\LARGE\nan} to get 5200 or above. 

\subsection{Concealed set of value tiles}
There are situations where melding with a cheap and slow hand may be acceptable. Recall that one of the reasons why melding is not worthwhile with a cheap and slow hand is that we will lose safety tiles if we meld. When that is less of your concern, melding may be an option even with a cheap and slow hand. 

%\bigskip
\begin{itembox}[r]{Concealed set of value tiles}
\bp
\wan{4}\wan{6}\wan{8}\tong{2}\tong{4}\suo{1}\suo{2}\suo{5}\suo{6}\suo{6}\zhong\zhong\zhong~~\xi\\
\hfill\footnotesize{{\jap Dora}~~~~~~~~~~}
\ep
\vspace{-20pt}The left player discarded {\LARGE\wan{5}} just now.
\end{itembox}

\noindent
This hand is both cheap and slow. Even after calling {\jap chii} on {\LARGE\wan{5}}, the hand is still 1-away with a bad wait. However, this hand has a concealed set of {\LARGE\zhong}, which can be used as three safe tiles when someone calls riichi. In such cases, you can meld as long as doing so advances the hand. That is, you can {\jap chii} any of {\LARGE\wan{5}\wan{7}\tong{3}\suo{3}}. You should not call {\jap chii} on {\LARGE\suo{4}-\suo{7}} or {\jap pon} on {\LARGE\suo{6}}, because doing so does not advance this hand from 2-away to 1-away or improve the wait/scores. 

\bigskip
\color{MyRed}
\begin{itembox}[c]{Melding judgement 4}\normalcolor
If you have a concealed set of value tiles, you can meld with a cheap and slow hand.
\end{itembox}\normalcolor

\subsection{When it is OK to meld with cheap \& slow hands}
There are a few more instances where melding with a cheap and slow hand is acceptable, summarized as follows. 

\be\itemsep.1pt
\i You are ahead of the game in South-4.\\
The hand value is not of your concern in such a situation. You can meld with a cheap hand; you can also meld even when melding significantly reduces the hand value.
\i There are two or more riichi bets on the table.\\
Winning any hand guarantees a minimum score of 3300 points in such a situation, as you get at least 1000 (your hand) + 2000 for riichi bets + 300 for continuation. This is not much different from winning a 3900 hand.
\i You are losing and you are the dealer.\\
You should aim for calling riichi as soon as possible in order to delay the opponents' attack. However, when you think you cannot make the hand ready for riichi soon enough, calling {\jap pon} or {\jap chii} early may serve the same purpose. 
\ee

\newpage

\section{Calling {\jap kan}} \label{sec:kong} \index{kan@{\jap kan} (kong)} \index{kantsu@{\jap kantsu} (quad)}
\index{quad@quad ({\jap kantsu})}

There are three ways to call {\jap kan} (kong) --- making a concealed quad ({\jap ankan}), making an open quad ({\jap daiminkan}), and extending an open set to an open quad ({\jap kakan}). I will discuss decision criteria for each of the three cases in turn.

\subsection{Concealed quad ({\jap ankan})}
The benefits of making a concealed quad includes:
\bi \itemsep.1em
\i another chance to draw a tile;
\i increased minipoints; and
\i possibilities of getting more {\jap dora}.
\ei
Conditions to justify making a concealed quad includes:
\bi \itemsep.1em
\i 1-away, where at least one block has a good wait;
\i 2-away, where all the blocks have a good wait;
\i you need more {\jap dora} or more minipoints to improve the placement (especially in South-4);
\i you are losing badly.
\ei

With this in mind, consider several examples. ({\jap Dora} is {\LARGE\xi} in all the examples.)

\bp
{\small (1)}
\wan{2}\wan{4}\wan{6}\tong{2}\tong{3}\tong{5}\tong{6}\tong{7}\suo{2}\suo{2}\bei\bei\bei\bei\\
{\small (2)}
\wan{3}\wan{4}\wan{4}\tong{2}\tong{3}\tong{5}\tong{6}\tong{7}\suo{2}\suo{2}\bei\bei\bei\bei\\
{\small (3)}
\wan{3}\wan{4}\wan{4}\tong{2}\tong{3}\tong{6}\tong{7}\tong{7}\suo{2}\suo{2}\bei\bei\bei\bei\\
{\small (4)}
\wan{3}\wan{5}\tong{1}\tong{3}\tong{5}\tong{6}\tong{7}\suo{2}\suo{2}\bei\bei\bei\bei\bai\\
{\small (5)}
\wan{3}\wan{5}\tong{1}\tong{1}\tong{3}\tong{6}\tong{6}\tong{7}\bei\bei\bei\bei\bai\bai\\
\ep

\noindent (1) This hand is 1-away, and one block has a good wait and another has a bad wait. You can call {\jap kan}. 

%\bigskip
\noindent (2) This is a perfect 1-away hand, so you can call {\jap kan}.

%\bigskip
\noindent (3) This is a perfect 2-away hand, so you can call {\jap kan}.

%\bigskip
\noindent (4) This hand is 1-away, but all the remaining blocks have a bad wait (closed wait). You should not call {\jap kan}.

\noindent (5) This hand is 2-away with a bad wait. You should not call {\jap kan} in normal situations. However, if you are in South-4, and you need 2000 points to win the game, then you should {\jap kan} immediately. As the hand will have at least 60 minipoints, you can get 2000 points with one {\jap han} (White Dragon). 

\bigskip
\begin{itembox}[c]{{\jap Kan} judgement 1}
In principle, your hand needs to be close to ready to justify making a concealed quad. 
\end{itembox}

\subsubsection{When not to make a concealed quad}

Calling {\jap kan} also comes with some cost, including:
\bi \itemsep.1em
\i you may lose safety tiles;
\i the new {\jap dora} may go to the opponents.
\ei
When the following conditions are present, you should refrain from calling {\jap kan}.
\bi \itemsep.1em
\i the hand is close to ready for {\jap chiitoitsu} as well;
\i one of the four tiles can be used a good floating tile;
\i you lose some {\jap yaku} if making the set into a quad.
\ei

\bigskip
\begin{itembox}[r]{Concealed quad?}
\bp
\wan{4}\wan{4}\wan{4}\wan{4}\wan{8}\wan{8}\tong{4}\tong{5}\tong{6}\tong{7}\suo{4}\suo{5}\suo{6}\suo{8}\\
\ep\index{sanshoku@{\jap sanshoku}}
\vspace{-10pt}It is your turn. What would you do?
\end{itembox}
\noindent
If you call {\jap kan}, the hand will be 1-away and the wait will not be terribly bad; it can be made ready if you draw any of \\{\LARGE\wan{8}\tong{2}\tong{3}\tong{4}\tong{5}\tong{6}\tong{7}\tong{8}\tong{9}\suo{3}\suo{6}\suo{7}\suo{9}}, and you may draw one of these tiles as a {\jap rinshan} tile (the bonus draw after {\jap kan}). However, the resulting hand will be either riichi only or riichi + {\jap tanyao} only, sometimes with a bad wait. 

\bigskip
If you choose not to call {\jap kan} and discard {\LARGE\suo{8}}, you can treat one of the four tiles of {\LARGE\wan{4}} as a floating tile that could form a side-wait protorun. The hand will be a side-wait ready hand if you draw any of {\LARGE\wan{3}\wan{5}\tong{3}\tong{5}\tong{6}\tong{8}}. Moreover, if you draw {\LARGE\wan{5}} or {\LARGE\wan{6}}, the hand will be ready for {\jap sanshoku} of 456. 
If you call {\jap kan}, on the other hand, the hand will lose the ability to accept {\LARGE\wan{2}\wan{3}\wan{5}\wan{6}} that would otherwise make the hand ready.
Therefore, you should not call {\jap kan} at this point and simply discard {\LARGE\suo{8}}. You can call {\jap kan} later if the hand becomes ready by drawing a {\jap pinzu} (dots) tile.

\bigskip
\begin{itembox}[r]{Concealed quad?}
\bp
\wan{4}\wan{4}\wan{4}\wan{4}\wan{7}\wan{8}\wan{9}\tong{4}\tong{5}\suo{1}\suo{2}\suo{3}\suo{7}\suo{8}\\
\ep
\vspace{-10pt}It is your turn. What would you do?
\end{itembox}
\noindent
If you call {\jap kan}, you will lose {\jap pinfu}. Moreover, if you draw a tile that completes one of the two side-wait protoruns after calling {\jap kan}, the hand becomes a single-wait hand. You should thus discard {\LARGE\wan{4}}. Then, if you complete one of the side-wait protoruns first, you can discard another {\LARGE\wan{4}} to make the hand ready for {\jap pinfu}. 

\bigskip
\begin{itembox}[c]{{\jap Kan} judgement 2}
There are cases where we should not make a concealed quad even when the hand is (close to) ready.
\end{itembox}


\subsection{Open quad ({\jap daiminkan})}
Conditions to justify making an open quad are a little bit more demanding than the conditions to justify making a concealed quad. 

\bigskip
You can make an open quad in any of the following situations:
\bi \itemsep.1em
\i the hand is ready with a good wait, and the hand value is between 2000 and 5200 points;
\i you need more {\jap dora} or minipoints to improve the placement (especially in South-4);
\i you are losing badly.
\ei

\bigskip
\begin{itembox}[r]{Open quad?}
\bp
\wan{6}\wan{7}\tong{8}\tong{8}\suo{1}\suo{2}\suo{3}\zhong\zhong\zhong~~\rtong{3}\tong{1}\tong{2}~\suo{2}\\
\hfill\footnotesize{{\jap Dora}~~~~~~~~}
\ep
\vspace{-20pt}The right player discarded {\LARGE\zhong} just now.
\end{itembox}
\noindent This ready hand is currently worth 2000 points with a good wait. Calling {\jap kan} on {\LARGE\zhong} is therefore justifiable. If any one of the tiles in your hand becomes {\jap dora}, the hand value increases from 2000 to 5200. If you get {\jap rinshan tsumo} in addition, it will be {\jap mangan}. 

\bigskip
\begin{itembox}[r]{Open quad?}
\bp
\wan{8}\wan{9}\tong{8}\tong{8}\suo{1}\suo{2}\suo{3}\zhong\zhong\zhong~~\rtong{3}\tong{1}\tong{2}~\suo{2}\\
\hfill\footnotesize{{\jap Dora}~~~~~~~~}
\ep
\vspace{-20pt}The right player discarded {\LARGE\zhong} just now.
\end{itembox}
\noindent This ready hand is currently 2000 points with a bad wait. Calling {\jap kan} on {\LARGE\zhong} is \emph{not} justifiable when the wait is bad.

\bigskip
\begin{itembox}[r]{Open quad?}
\bp
\wan{2}\wan{3}\tong{8}\tong{8}\suo{1}\suo{2}\suo{3}\zhong\zhong\zhong~~\rtong{3}\tong{1}\tong{2}~\suo{9}\\
\hfill\footnotesize{{\jap Dora}~~~~~~~~}
\ep
\vspace{-20pt}The right player discarded {\LARGE\zhong} just now.
\end{itembox}
\noindent This ready hand is currently 1000 points with a good wait. Calling {\jap kan} on {\LARGE\zhong} is \emph{not} justifiable. Even when one of the tiles in your hand becomes new {\jap dora}, the hand value only increases from 1000 to 2600 points. 

\subsection{From an open set to an open quad ({\jap kakan})}
When you draw the fourth tile of an open set, you have an opportunity to extend the open set to an open quad. Conditions to justify extending an open set to a quad are more demanding than those for a concealed quad but less demanding than those for a regular open quad. 
Doing so is less foolhardy compared with a regular open quad because you are not losing four safety tiles. At the same time, this is riskier than making a concealed quad because you may be running the risk of getting {\jap chankan} (Robbing the Kong). 

\bigskip
You can extend an open set to an open quad in any of the following situations:
\bi \itemsep.1em
\i the hand is 1-away or better with a good wait, and it has two {\jap han} or more;
\i the hand is 1-away or better, and there are not many turns left to draw tiles;
\i you need more {\jap dora} or minipoints to improve the placement (especially in South-4);
\i you are losing badly.
\ei

\bigskip
\begin{itembox}[r]{Open set to quad?}
\bp
\wan{2}\wan{3}\tong{1}\tong{1}\tong{1}\tong{4}\tong{4}\suo{2}\suo{2}\suo{3}~\zhong\rzhong\zhong~\wan{3}\\
\hfill\footnotesize{{\jap Dora}~~~~~~~~~}
\ep
\vspace{-20pt}You drew the fourth {\LARGE\zhong} just now.
\end{itembox}
\noindent This 1-away hand has two {\jap han} and a good wait. Calling {\jap kan} on {\LARGE\zhong} is justifiable. 

\bigskip
\begin{itembox}[r]{Open set to quad?}
\bp
\wan{2}\wan{3}\tong{1}\tong{1}\tong{4}\tong{5}\suo{2}\suo{3}\suo{9}\suo{9}~\zhong\rzhong\zhong~\tong{1}\\
\hfill\footnotesize{{\jap Dora}~~~~~~~~~}
\ep
\vspace{-20pt}You drew the fourth {\LARGE\zhong} just now.
\end{itembox}
\noindent This hand has three {\jap han} and a good wait. However, since it is 2-away from ready, calling {\jap kan} on {\LARGE\zhong} is \emph{not} justifiable. 

\newpage
\section{Miscellaneous tips for melding}

\subsection{Think ahead}

When you call {\jap pon}, you have to say ``{\jap pon}! \textipa{[p\'\textopeno\ng]}'' out loud immediately and nothing else. There is no such call as ``Wait!'',  and you will have to forgo your call if (1) the next player has already drawn their tile before you call {\jap pon} or (2) another player has already called {\jap chii} before you do.\footnote{A {\jap pon} call takes precedence over a {\jap chii} call, but only if calls are made simultaneously. If the {\jap chii} call was made well before the {\jap pon} call, the {\jap chii} call should take precedence.}
This means that you need to think ahead and make up your mind about what tile to call \emph{before} the tile is discarded. That is, you should think about what tile(s) can improve the wait and/or the scores of your hand all the time. For example, consider the following hand. 

\begin{itembox}[r]{Thinking ahead}
\bp
\wan{2}\wan{4}\wan{5}\wan{5}\wan{5}\tong{1}\tong{2}\tong{3}\tong{4}\tong{4}~~\bai\bai\rbai
\ep
\vspace{-10pt}
What tile(s) are you waiting for?
\end{itembox}
The hand is ready, waiting for {\LARGE\wan{3}}. However, you should also be prepared for melding further to improve the wait and/or the scores. 
If you draw or call {\jap pon} on {\LARGE\tong{4}} and discard {\LARGE\wan{2}}, the wait will be upgraded to an irregular 3-way wait of {\LARGE\wan{3}-\wan{6} \wan{4}}. Moreover, if you draw or call {\jap pon} on the red {\LARGE\rfw} and discard {\LARGE\wan{2}}, not only the scores get better but also the wait will be upgraded to a side wait of {\LARGE\wan{3}-\wan{6}}.

\bigskip
Relatedly, think about what to discard upon melding \emph{before} you call. 
If you are unsure about what to discard upon melding, it probably means you should not make the call. 

\subsection{Be ready for {\jap dora}}
You should also think about how to utilize {\jap dora} when melding. Consider the following hand. 

\begin{itembox}[r]{Utilizing {\jap dora}}
\bp
\wan{8}\wan{8}\tong{4}\tong{5}\tong{6}\tong{7}\tong{8}\suo{2}\suo{3}\suo{3}~\fa\rfa\fa~~\tong{5}\\
\hfill\footnotesize{{\jap Dora}~~~~~~~~}
\ep
\vspace{-20pt}
The left player discarded {\LARGE\tong{6}} just now.
\end{itembox}
\noindent We should definitely call {\jap chii} on {\LARGE\tong{6}} to make the hand ready, but the question here is: should we {\jap chii} with {\LARGE\tong{4}\tong{5}} or with {\LARGE\tong{7}\tong{8}}? Let's compare the resulting hands in each of the two possibilities. 
\bp
\wan{8}\wan{8}\tong{6}\tong{7}\tong{8}\suo{2}\suo{3}~\rtong{6}\tong{4}\tong{5}~\fa\rfa\fa~\tong{5}\\
\wan{8}\wan{8}\tong{4}\tong{5}\tong{6}\suo{2}\suo{3}~\rtong{6}\tong{7}\tong{8}~\fa\rfa\fa~\tong{5}\\
\hfill\footnotesize{{\jap Dora}~~~~~~~~~~}
\ep
Notice that the first hand can accept another {\jap dora}. That is, if you draw {\LARGE\tong{5}}, you can keep it and discard {\LARGE\tong{8}} to improve the hand value from 2000 to 3900.
With the second hand, you will have to discard the {\jap dora} when you draw another. 

\bigskip
Calling {\jap chii} with {\LARGE\tong{4}\tong{5}} is better also from a perspective of defense. Having to discard {\LARGE\tong{8}} against an opponent's riichi is much better than having to discard {\LARGE\tong{5}}. {\LARGE\tong{5}} can be captured by both {\LARGE\tong{2}-\tong{5}} and {\LARGE\tong{5}-\tong{8}} {\jap suji}, whereas {\LARGE\tong{8}} can be captured only by {\LARGE\tong{5}-\tong{8}} {\jap suji}. Moreover, even if you deal into an opponent's {\LARGE\tong{5}-\tong{8}} wait, the hand value would be lower if you discard {\LARGE\tong{8}} than {\LARGE\tong{5}}, on average.\footnote{Of course, this is unless an opponent has {\jap sanshoku} of 678.} 


\subsection{Be mindful of the seating}
Each time you take a tile from the facing player ({\jap toimen}) with a {\jap pon} (or {\jap kan}) call, the left player's ({\jap kamicha}) turn gets skipped. Likewise, each time you take a tile from the right player ({\jap shimocha}) with {\jap pon/kan}, the facing player's turn gets skipped. At the same time, the right player will have an additional chance to draw a tile in either of the two instances. In this sense, your act of calling {\jap pon/kan} benefits the right player while penalizing the left and the facing players. 
It is useful to keep this in mind in making a melding choice, especially when the benefit of melding only slightly outweighs the benefit of keeping the hand closed in terms of tile efficiency.

\bigskip
For example, when you are North, you should not meld as aggressively because doing so will benefit the dealer. Likewise, when you are South, you should try to call {\jap pon} from the facing player (North) rather than from the left player (East), so that you can penalize the dealer. 
The same is true when there is a clear front-runner in the game. When your right player is much ahead of the game, you should try to have a closed hand rather than a melded hand. On the other hand, when your left player is leading the game, you should meld a bit more aggressively so you can penalize him. 

\begin{itembox}[c]{Seating-related tip 1}
When your right player is the dealer and/or the front-runner, try not to call {\jap pon} too much. 
\end{itembox}

\bigskip
Applying the same logic, you do not want your right player to call {\jap pon} from your left player. This means that, if you plan to discard something that can be {\jap pon}'ed by the right player, you should do so sooner rather than later. For example, suppose you are East, and you are deciding which one of the three valueless wind tiles {\LARGE\nan\xi\bei} to discard in the 1st turn. In this case, you should discard {\LARGE\nan} first. If the South player calls {\jap pon} on {\LARGE\nan} you discard, that would be much better than if he called {\jap pon} from the North player. Moreover, if you discard {\LARGE\nan}, there is a good chance that another player may do the same in the 1st turn as well, lowering the chance that the South player builds a pair of {\LARGE\nan} in later turns and calls {\jap pon}. 

\begin{itembox}[c]{Seating-related tip 2}
When discarding valueless wind tiles, discard the right player's wind first, then the facing player's wind next.
\end{itembox}
	\index{valueless wind@valueless wind ({\jap otakaze})!how to discard}
	\index{otakaze@{\jap otakaze} (valueless wind)!how to discard}



